\documentclass[a4paper,12pt]{article}
\usepackage{multicol}

\usepackage[T1]{fontenc}
\usepackage{graphicx}
\usepackage[english]{babel}
\usepackage{amsmath,amsfonts,amsthm}

\usepackage[a4paper]{geometry}
\linespread{1.3}

\begin{document}

\noindent\large\textbf{\underline{Solutions to questions from week 1}}

\vspace{0.5cm}

\noindent\normalsize\textbf{Video 1}

 
%\begin{minipage}{\textwidth}
%\hspace{-0.78cm}

\noindent 1. -Computer operations are synchronized to a clock. Brain operations happen in real time. 

-Brains appear to consume much less energy than computers in relation to their processing capacity. 

-Brain signals are analytical and noisy, computer signals are digital and deterministic.

%\end{minipage}

\noindent 2. Estimated to be around $10^{11}$.

\noindent 3. Machine learning branched from "computational neuroscience" in the 90s by taking distance from biological realism. In the 2010s, the capacities of artificial neural networks (rebranded as "deep learning") started being re-explored in numerous applications, feeding from both "computational neuroscience" and "machine learning" research.

\noindent 4. - Statistical analysis of experimental data.

- Computational modelling of the brain.

\vspace{0.5cm}

\noindent\normalsize\textbf{Video 2}

\noindent 1. 2\% // 20\%

\noindent 2. -Long-term memory.

- Spatial navigation.

\noindent 3. Basal ganglia.

\noindent 4. The cerebellum.

\noindent 5. Basal ganglia.

\noindent 6. That there is a correspondence in the way different parts of the body are represented in different specific parts of the homunculi.


\vspace{0.5cm}

\noindent\normalsize\textbf{Video 3}

\noindent 1. Axons

\noindent 2. The axons can be very long because they reach distant brain areas.

\noindent 3. Axons = from the soma.

Dendrites = to the soma.

\noindent 4. The synapses.


\vspace{0.5cm}

\noindent\normalsize\textbf{Video 4}

\noindent 1. Let's consider a generic scenario in which each synapse triggers a post-synaptic potential of 1mV, and the spiking threshold of the post-synaptic neuron is 20mV above the resting state. In this case you would need \emph{20} of these excitatory synapses arriving synchronously to be able to induce a spike.

\noindent 2. How strongly the synapse affects the postsynaptic neuron's membrane potential.

\noindent 3. The spiking threshold.

\noindent 4. Recurrent. The signal can be repropagated repeatedly within the network, in a sort of short-termed memory, enabling more complex operations. 

\vspace{0.5cm}

\noindent\normalsize\textbf{Video 5}

\noindent 1. About one kiloHertz. That is because an action potential can be as short as one milisecond.

\noindent 2. \emph{in vivo} = inside the live animal

\emph{in vitro} = in a sample that has been removed from the animal

\noindent 3. Intracellular recordings can register subthreshold fluctuations of the membrane voltage.

\noindent 4. It measures changes in the blood-oxygen levels (BOLD signal). The levels of blood supply are correlated to neural activity, but vary in a slower scale.

\noindent 5. With Calcium imaging it is possible to directly see the spatial distribution of the neurons. 

\end{document}
